\documentclass[12pt,a4paper]{article} % classe article, taille 12, papier A4
\usepackage{fontspec} % permet de gérer la police

\usepackage{graphicx} % gère les images
\usepackage{array} % gère les tableaux
\usepackage{hyperref} % gère les liens
\usepackage{csquotes} % gère les citations et les guillemets

\usepackage{polyglossia} % gère l'aspect multilingue des documents
\setmainlanguage{french} % sélection de la langue principale du document

\usepackage[citestyle=verbose]{biblatex} % appel de biblatex, de nombreuses autres options disponibles
\bibliography{web} % appel du fichier .bib sans l'extension

% emplacement pour d'autres packages

\title{\textbf{L’IMPACT TRANSFORMATEUR DU WEB 2.0 : INTERACTION, INNOVATION ET OPPORTUNITÉS POUR LES ENTREPRISES} } % le titre du document
\author{AZOSSIE TERESA FABIOLA} % son auteur
\date{Janvier 2024} % la date


%% début du document

\begin{document} % début du corps du texte
	
	\maketitle % affiche le titre, l'auteur et la date
	\begin{figure}[h]
		
		\centering
		
		\includegraphics[scale=1]{ulb-logo}
		
	\end{figure}
	\tableofcontents % affiche une table des matières correspondant aux sections du document
	
	\section{Introduction}
	L'évolution d'Internet a marqué des étapes significatives depuis ses débuts. De simples pages statiques sont progressivement devenues des espaces interactifs, ouvrant la voie à ce que l'on appelle communément le web2.0. Cette mutation a engendré une transformation majeure dans la façon dont les individus interagissent, communiquent et travaillent en ligne. Cette dissertation se propose d'explorer en profondeur l'émergence, les caractéristiques et l'impact du web2.0. Le terme "web2.0" a été popularisé au début des années 2000 pour décrire cette nouvelle ère d'Internet. Cette évolution n'est pas simplement une question de technologie, mais plutôt une révolution sociale et culturelle. Elle a instauré une ère où les utilisateurs ont acquis un rôle central, passant de simples consommateurs de contenu à des contributeurs actifs et engagés dans la création et le partage d'informations. Afin de mieux appréhender cette évolution, il est nécessaire de retracer l'histoire du web2.0, d'en définir les contours et de le situer par rapport aux précédentes versions d'Internet. Par la suite, cette dissertation explorera en détail les caractéristiques spécifiques du web2.0 ainsi que son utilité grandissante pour les entreprises. Enfin, une réflexion sera proposée sur les implications actuelles et futures de cette révolution numérique. À travers cette analyse approfondie du web2.0, nous serons en mesure de comprendre pleinement son impact sur nos vies quotidiennes, les entreprises et la société dans son ensemble.
	\section{\textbf{HISTOIRE DU WEB 2.0 : ORIGINE ET DÉFINITION}}
	
	\subsection{\textbf{ORIGINE DU WEB2.0}}
	Le concept du web2.0 a commencé à émerger vers le milieu des années 2000, marquant une transition majeure dans l'évolution d'Internet. Il est né d'une convergence de facteurs technologiques, sociaux et culturels. Des entreprises telles que Google, Amazon et eBay ont contribué à façonner cette nouvelle ère en introduisant des services interactifs et en mettant l'accent sur la participation des utilisateurs.
	
	Ce changement s'est opéré à travers une série d'innovations telles que le développement des réseaux sociaux, des blogs, des plateformes de partage de contenu multimédia comme YouTube et Flickr, ainsi que des outils de collaboration en ligne tels que les wikis. Ces nouvelles fonctionnalités ont donné aux utilisateurs la possibilité de créer, partager et collaborer de manière inédite.
	
	
	\subsection{\textbf{DÉFINITION DU WEB2.0}}
	Définir le web2.0 est complexe car il ne se limite pas à une technologie spécifique mais englobe un ensemble de principes. Il s'agit d'une évolution vers un Internet axé sur l'interaction, la collaboration et la participation des utilisateurs. Le web2.0 se caractérise par une rupture avec le modèle précédent où les sites étaient statiques et où la contribution des utilisateurs était limitée.
	
	Au cœur du web2.0 se trouve l'idée de "partage". Les utilisateurs peuvent non seulement consommer du contenu mais aussi le créer, le modifier et le partager facilement. Cette transformation a entraîné un changement fondamental dans la manière dont les individus interagissent en ligne, favorisant la création de communautés et l'échange d'idées à une échelle sans précédent.
	
	Cette évolution vers le web2.0 a été caractérisée par un passage du modèle centré sur l'entreprise à un modèle centré sur l'utilisateur, où la contribution de chacun est valorisée et intégrée dans la construction même du web.
	
	Cette partie met en lumière l'émergence et la définition du web2.0, soulignant l'évolution significative qu'il a apportée à la façon dont les individus interagissent avec Internet.
	
	
	\section{\textbf{LA DIFFÉRENCE ENTRE LE WEB 1.O, LE WEB 2.0 ET LE WEB 3.0}}
	\subsection{\textbf{LE WEB 1.0 : Le Web Statique}}
	Au commencement, le Web 1.0 était un espace essentiellement statique où l'information était unilatéralement diffusée par des entreprises et des institutions. Comme le souligne Tim O’Reilly, pionnier du terme "web2.0", le Web 1.0 était "lire seulement", les utilisateurs n'étaient que des consommateurs passifs de contenu (O’Reilly, 2005).  
	Les sites web étaient principalement des pages statiques offrant des informations sans interaction significative. Un exemple classique serait les sites web de présentation d'entreprises, où les utilisateurs ne pouvaient que consulter les données fournies, sans possibilité de contribution ou de rétroaction. Au départ, Internet n’était pas aussi développé qu’il l’est aujourd’hui. Il était même assez limité. À tel point, que beaucoup pensaient qu’il n’aurait pas d’avenir. Les internautes qui utilisaient cette nouvelle plateforme pouvaient partager des informations à un large public (par exemple : présenter leur entreprise). Néanmoins, l’information publiée sur le web était figée sans aucune réaction ni commentaire des autres utilisateurs et tout le monde n’avait pas la possibilité de publier de l’information. La majorité des internautes pouvait seulement consulter. C’est ce qu’on a appelé le Web 1.0. Cependant, Internet a su contourner cette limite et a connu un succès fulgurant en quelques années et s’est imposé dans le quotidien de tous.
	Selon Tim Berners-Lee, le père du World Wide Web, le Web 1.0 était une phase où les utilisateurs étaient principalement des consommateurs passifs d'informations sans possibilité significative d'interaction ou de contribution (Berners-Lee, 1999). Par exemple, les sites web d'alors étaient principalement des pages statiques offrant des données sans interaction réelle.
	\subsection{\textbf{LE WEB 2.0 : L’ère De L'interaction Et De La Participation}}
	En contraste, le Web 2.0 est une évolution marquée par une interaction dynamique et une participation active des utilisateurs. Il se caractérise par une multitude de plateformes interactives telles que les réseaux sociaux, les blogs et les wikis. Comme l'affirme Clay Shirky, théoricien des médias numériques, le web2.0 est "write-read" permettant aux utilisateurs de créer et partager du contenu de manière collaborative (Shirky, 2003). Les utilisateurs deviennent des contributeurs, générant du contenu, partageant des idées, et interagissant avec d'autres utilisateurs à travers des commentaires, des partages et des contributions sur les réseaux sociaux. Cette ère a vu l'émergence de sites comme Wikipédia, où des milliers de contributeurs collaborent pour créer une encyclopédie mondiale en ligne.
	Le Web 2.0, c’est celui que nous connaissons tous aujourd’hui et que nous utilisons quotidiennement. C’est le Web de la création de contenu. C’est celui qui permet de partager toutes sortes d’informations et de données entre utilisateurs. Désormais, avec le web 2.0, les utilisateurs peuvent à leur tour créer du contenu et les partager avec d’autres internautes. C’est l’arrivée des blogs et des réseaux sociaux ! Cependant, qui dit croissance dit aussi dérive et le premier problème du Web 2.0 c’est qu’il a fait perdre aux utilisateurs la main sur la gestion de leurs données personnelles.
	Clay Shirky, dans son ouvrage "Here Comes Everybody : The Power of Organizing Without Organizations", souligne que le Web 2.0 a donné naissance à une ère où les utilisateurs sont devenus des contributeurs actifs, générant du contenu et interagissant sur les plateformes sociales et collaboratives (Shirky, 2008). Un exemple remarquable serait l'émergence de Wikipédia où des milliers de contributeurs collaborent pour construire une encyclopédie en ligne.
	\subsection{\textbf{LE WEB 3.0 : Vers l'Internet Sémantique et Contextuel}}
	Le Web 3.0, souvent évoqué comme l'Internet sémantique ou l'Internet des objets, représente une nouvelle étape dans l'évolution d'Internet. Il se concentre sur l'intelligence artificielle, la contextualisation des données et l'interopérabilité des systèmes.
	Cet Internet du futur vise à créer un environnement où les machines peuvent comprendre et interpréter le contenu, offrant ainsi une expérience plus personnalisée et contextualisée aux utilisateurs. Des exemples incluent les assistants virtuels comme Siri d'Apple ou Alexa d'Amazon, qui utilisent l'IA pour comprendre et répondre aux requêtes des utilisateurs de manière plus contextuelle. Cette comparaison met en lumière l'évolution significative du Web, du statique Web 1.0 à l'interactif Web 2.0, et la perspective prometteuse du Web 3.0 axé sur l'IA et la contextualisation des données. 
	Tim Berners-Lee, également auteur sur cette phase émergente, parle de l'Internet sémantique dans son livre "Weaving the Web : The Original Design and Ultimate Destiny of the World Wide Web" (Berners-Lee, 1999). Il évoque la vision d'un Web où les machines comprennent le contenu et offrent des expériences plus contextuelles aux utilisateurs. Par exemple, les assistants vocaux comme Siri ou Alexa illustrent cette tendance vers un Internet qui s'adapte et comprend les besoins de l'utilisateur.
	
	%% bibliographie
	\printbibliography
	
\end{document}