\documentclass[12pt,a4paper]{article} % classe article, taille 12, papier A4
\usepackage{fontspec} % permet de gérer la police

\usepackage{graphicx} % gère les images
\usepackage{array} % gère les tableaux
\usepackage{hyperref} % gère les liens
\usepackage{csquotes} % gère les citations et les guillemets

\usepackage{polyglossia} % gère l'aspect multilingue des documents
\setmainlanguage{french} % sélection de la langue principale du document

\usepackage[citestyle=verbose]{biblatex} % appel de biblatex, de nombreuses autres options disponibles
\bibliography{web} % appel du fichier .bib sans l'extension

% emplacement pour d'autres packages

\title{\textbf{L’IMPACT TRANSFORMATEUR DU WEB 2.0 : INTERACTION, INNOVATION ET OPPORTUNITÉS POUR LES ENTREPRISES} } % le titre du document
\author{AZOSSIE TERESA FABIOLA} % son auteur
\date{Janvier 2024} % la date


%% début du document

\begin{document} % début du corps du texte
	
	\maketitle % affiche le titre, l'auteur et la date
	\begin{figure}[h]
		
		\centering
		
		\includegraphics[scale=1]{ulb-logo}
		
	\end{figure}
	\tableofcontents % affiche une table des matières correspondant aux sections du document
	
	\section{Introduction}
	L'évolution d'Internet\footnote[1]{réseau mondial d'ordinateurs connectés permettant le partage d'informations.} a marqué des étapes significatives depuis ses débuts. De simples pages statiques sont progressivement devenues des espaces interactifs, ouvrant la voie à ce que l'on appelle communément le web2.0. Cette mutation a engendré une transformation majeure dans la façon dont les individus interagissent, communiquent et travaillent en ligne. Cette dissertation se propose d'explorer en profondeur l'émergence, les caractéristiques et l'impact du web2.0. Le terme "web2.0" a été popularisé au début des années 2000 pour décrire cette nouvelle ère d'Internet. Cette évolution n'est pas simplement une question de technologie, mais plutôt une révolution sociale et culturelle. Elle a instauré une ère où les utilisateurs ont acquis un rôle central, passant de simples consommateurs de contenu à des contributeurs actifs et engagés dans la création et le partage d'informations. Afin de mieux appréhender cette évolution, il est nécessaire de retracer l'histoire du web2.0, d'en définir les contours et de le situer par rapport aux précédentes versions d'Internet. Par la suite, cette dissertation explorera en détail les caractéristiques spécifiques du web2.0 ainsi que son utilité grandissante pour les entreprises. Enfin, une réflexion sera proposée sur les implications actuelles et futures de cette révolution numérique. À travers cette analyse approfondie du web2.0, nous serons en mesure de comprendre pleinement son impact sur nos vies quotidiennes, les entreprises et la société dans son ensemble.
	\section{\textbf{HISTOIRE DU WEB 2.0 : ORIGINE ET DÉFINITION}}
	
	\subsection{\textbf{ORIGINE DU WEB2.0}}
	Le concept du web2.0 a commencé à émerger vers le milieu des années 2000, marquant une transition majeure dans l'évolution d'Internet. Il est né d'une convergence de facteurs technologiques, sociaux et culturels. Des entreprises telles que Google, \href{https://www.amazon.com.be/?tag=betxtabkgodef-21&ref=pd_sl_7q4ld6cerg_e&adgrpid=162532280388&hvpone=&hvptwo=&hvadid=675303268037&hvpos=&hvnetw=g&hvrand=3781438560409787019&hvqmt=e&hvdev=c&hvdvcmdl=&hvlocint=1001011&hvlocphy=1005402&hvtargid=kwd-10573980&hydadcr=24879_2326196&language=fr_BE}{Amazon} et eBay ont contribué à façonner cette nouvelle ère en introduisant des services interactifs et en mettant l'accent sur la participation des utilisateurs.
	
	Ce changement s'est opéré à travers une série d'innovations telles que le développement des réseaux sociaux, des blogs, des plateformes de partage de contenu multimédia comme \href{https://www.youtube.com/}{YouTube} et Flickr, ainsi que des outils de collaboration en ligne tels que les wikis. Ces nouvelles fonctionnalités ont donné aux utilisateurs la possibilité de créer, partager et collaborer de manière inédite.
	
	
	\subsection{\textbf{DÉFINITION DU WEB2.0}}
	Définir le web2.0 est complexe car il ne se limite pas à une technologie spécifique mais englobe un ensemble de principes. Il s'agit d'une évolution vers un Internet axé sur l'interaction, la collaboration et la participation des utilisateurs. Le web2.0 se caractérise par une rupture avec le modèle précédent où les sites étaient statiques et où la contribution des utilisateurs était limitée.
	
	Au cœur du web2.0 se trouve l'idée de "partage". Les utilisateurs peuvent non seulement consommer du contenu mais aussi le créer, le modifier et le partager facilement. Cette transformation a entraîné un changement fondamental dans la manière dont les individus interagissent en ligne, favorisant la création de communautés et l'échange d'idées à une échelle sans précédent.
	
	Cette évolution vers le web2.0 a été caractérisée par un passage du modèle centré sur l'entreprise à un modèle centré sur l'utilisateur, où la contribution de chacun est valorisée et intégrée dans la construction même du web.
	
	Cette partie met en lumière l'émergence et la définition du web2.0, soulignant l'évolution significative qu'il a apportée à la façon dont les individus interagissent avec Internet.
	
	
	\section{\textbf{LA DIFFÉRENCE ENTRE LE WEB 1.O, LE WEB 2.0 ET LE WEB 3.0}}
	\subsection{\textbf{LE WEB 1.0 : Le Web Statique}}
	Au commencement, le Web 1.0 était un espace essentiellement statique où l'information était unilatéralement diffusée par des entreprises et des institutions. Comme le souligne Tim O’Reilly, pionnier du terme "web2.0", le Web 1.0 était \textit{"lire seulement"}, les utilisateurs n'étaient que des consommateurs passifs de contenu (O’Reilly, 2005).  
	Les sites web étaient principalement des pages statiques offrant des informations sans interaction significative. Un exemple classique serait les sites web de présentation d'entreprises, où les utilisateurs ne pouvaient que consulter les données fournies, sans possibilité de contribution ou de rétroaction. Au départ, Internet n’était pas aussi développé qu’il l’est aujourd’hui. Il était même assez limité. À tel point, que beaucoup pensaient qu’il n’aurait pas d’avenir. Les internautes qui utilisaient cette nouvelle plateforme pouvaient partager des informations à un large public (par exemple : présenter leur entreprise). Néanmoins, l’information publiée sur le web était figée sans aucune réaction ni commentaire des autres utilisateurs et tout le monde n’avait pas la possibilité de publier de l’information. La majorité des internautes pouvait seulement consulter. C’est ce qu’on a appelé le Web 1.0. Cependant, Internet a su contourner cette limite et a connu un succès fulgurant en quelques années et s’est imposé dans le quotidien de tous.
	Selon Tim Berners-Lee, le père du www\footnote[2]{Le world wide web est un système d'informations sur internet permettant d'accéder à des pages web et de naviguer à travers elles à l'aide de navigateurs.}, le Web 1.0 était une phase où les utilisateurs étaient principalement des consommateurs passifs d'informations sans possibilité significative d'interaction ou de contribution (Berners-Lee, 1999). Par exemple, les sites web d'alors étaient principalement des pages statiques offrant des données sans interaction réelle.
	\subsection{\textbf{LE WEB 2.0 : L’ère De L'interaction Et De La Participation}}
	En contraste, le Web 2.0 est une évolution marquée par une interaction dynamique et une participation active des utilisateurs. Il se caractérise par une multitude de plateformes interactives telles que les réseaux sociaux, les blogs et les wikis. Comme l'affirme Clay Shirky, théoricien des médias numériques, le web2.0 est "write-read" permettant aux utilisateurs de créer et partager du contenu de manière collaborative (Shirky, 2003). Les utilisateurs deviennent des contributeurs, générant du contenu, partageant des idées, et interagissant avec d'autres utilisateurs à travers des commentaires, des partages et des contributions sur les réseaux sociaux. Cette ère a vu l'émergence de sites comme Wikipédia, où des milliers de contributeurs collaborent pour créer une encyclopédie mondiale en ligne.
	Le Web 2.0, c’est celui que nous connaissons tous aujourd’hui et que nous utilisons quotidiennement. C’est le Web de la création de contenu. C’est celui qui permet de partager toutes sortes d’informations et de données entre utilisateurs. Désormais, avec le web 2.0, les utilisateurs peuvent à leur tour créer du contenu et les partager avec d’autres internautes. C’est l’arrivée des blogs et des réseaux sociaux ! Cependant, qui dit croissance dit aussi dérive et le premier problème du Web 2.0 c’est qu’il a fait perdre aux utilisateurs la main sur la gestion de leurs données personnelles.
	Clay Shirky, dans son ouvrage "Here Comes Everybody : The Power of Organizing Without Organizations", souligne que le Web 2.0 a donné naissance à une ère où les utilisateurs sont devenus des contributeurs actifs, générant du contenu et interagissant sur les plateformes sociales et collaboratives (Shirky, 2008). Un exemple remarquable serait l'émergence de Wikipédia où des milliers de contributeurs collaborent pour construire une encyclopédie en ligne.
	\subsection{\textbf{LE WEB 3.0 : Vers l'Internet Sémantique et Contextuel}}
	Le Web 3.0, souvent évoqué comme l'Internet sémantique ou l'Internet des objets, représente une nouvelle étape dans l'évolution d'Internet. Il se concentre sur l'intelligence artificielle, la contextualisation des données et l'interopérabilité des systèmes.
	Cet Internet du futur vise à créer un environnement où les machines peuvent comprendre et interpréter le contenu, offrant ainsi une expérience plus personnalisée et contextualisée aux utilisateurs. Des exemples incluent les assistants virtuels comme Siri d'Apple ou Alexa d'Amazon, qui utilisent l'IA\footnote[3]{l'intelligence artificielle est une discipline de l'informatique qui se concentre sur la création de machines capables d'imiter des fonctions cognitives humaines telles que l'apprentissage, le raisonnement et la résolution des problèmes} pour comprendre et répondre aux requêtes des utilisateurs de manière plus contextuelle. Cette comparaison met en lumière l'évolution significative du Web, du statique Web 1.0 à l'interactif Web 2.0, et la perspective prometteuse du Web 3.0 axé sur l'IA et la contextualisation des données. 
	Tim Berners-Lee, également auteur sur cette phase émergente, parle de l'Internet sémantique dans son livre "Weaving the Web : The Original Design and Ultimate Destiny of the World Wide Web" (Berners-Lee, 1999). Il évoque la vision d'un Web où les machines comprennent le contenu et offrent des expériences plus contextuelles aux utilisateurs. Par exemple, les assistants vocaux comme Siri\footnote[4]{l'assistant vocale d'Apple} ou Alexa\footnote[5]{l'assistant vocale d'Amazon} illustrent cette tendance vers un Internet qui s'adapte et comprend les besoins de l'utilisateur.
	\section{\textbf{LES CARACTÉRISTIQUES DU WEB 2.0 ET SON UTILITÉ AU SEIN DES ENTREPRISES }}
	\subsection{\textbf{LES CARACTÉRISTIQUES DU WEB 2.0}}
	
	Le Web 2.0 est défini par plusieurs caractéristiques distinctives qui ont révolutionné l'expérience en ligne :
	\begin{itemize}
		\item \textbf{Interactivité accrue}
		Le Web 2.0 permet une interaction bidirectionnelle entre les utilisateurs et les plateformes en ligne. Les réseaux sociaux, les commentaires sur les blogs et les forums illustrent cette interactivité. Comme le souligne Dan Gillmor dans "We the Media", cette participation active transforme les consommateurs passifs en acteurs de la création de contenu (Gillmor, 2004).
		\item \textbf{Collaboration et Partage}
		Une caractéristique clé du Web 2.0 est la capacité des utilisateurs à collaborer et partager du contenu. Les plateformes comme YouTube, permettant de publier et de partager des vidéos, ou Wikipédia, construite par la collaboration de milliers de personnes, démontrent cette tendance collaborative (Shirky, 2008).
		\item \textbf{Contenu riche et diversifié}
		Le Web 2.0 offre une variété de formats de contenu allant au-delà du simple texte. Des médias tels que des images, des vidéos, des podcasts et des infographies enrichissent l'expérience utilisateur, favorisant une communication plus visuelle et immersive (O'Reilly, 2005).
		\item \textbf{Accessibilité omniprésente}
		Les plateformes du Web 2.0 sont accessibles via différents appareils et systèmes, offrant une expérience utilisateur uniforme. Les smartphones, tablettes et ordinateurs de bureau peuvent tous accéder aux mêmes fonctionnalités et contenus, favorisant une connectivité globale (Tapscott et Williams, 2006).
		\item\textbf{Utilisation des Technologies AJAX et HTML5}
		Le Web 2.0 a vu l'émergence de technologies telles qu'AJAX (Asynchronous JavaScript and XML) et HTML5, permettant des interactions plus fluides et dynamiques sur les sites web. Cette évolution a amélioré l'expérience utilisateur en permettant des mises à jour en temps réel et des interfaces plus réactives (O'Reilly, 2005).
		\item \textbf{Personnalisation et Recommandations}
		Les plateformes du Web 2.0 offrent souvent des fonctionnalités de personnalisation avancées. Des algorithmes de recommandation basés sur les préférences des utilisateurs sont largement utilisés pour proposer du contenu ciblé, améliorant ainsi l'expérience individuelle de chaque utilisateur (Anderson, 2006).
		\item \textbf{Émergence des Réseaux Sociaux}
		L'un des aspects les plus emblématiques du Web 2.0 est l'apparition des réseaux sociaux. Des plateformes comme Facebook, Twitter, et \href{https://fr.linkedin.com/}{LinkedIn} ont remodelé la manière dont les individus interagissent en ligne, favorisant la connexion entre les individus à travers le monde et la création de communautés virtuelles (Boyd et Ellison, 2007).
		\item \textbf{Plateformes de Crowdsourcing}
		Le Web 2.0 a encouragé le développement de plateformes de crowdsourcing\footnote[6]{collaboration en ligne à grande échelle} telles que Kickstarter ou Threadless, permettant à des communautés en ligne de financer des projets créatifs ou de participer à la conception de produits. Ces plateformes illustrent la capacité du Web 2.0 à mobiliser collectivement les ressources et les idées des individus (Brabham, 2008).
	\end{itemize}
	Ces caractéristiques ont transformé la façon dont les individus interagissent avec Internet, passant d'une consommation passive de contenu à une participation active et à une création collaborative. Elles ont également remodelé la manière dont les entreprises interagissent avec leurs clients et gèrent leurs stratégies de communication en ligne. Ces caractéristiques du Web 2.0 ont considérablement élargi les possibilités en ligne, transformant Internet en un espace interactif, collaboratif et personnalisé, offrant de nouvelles opportunités tant pour les individus que pour les entreprises.
	
	\subsection{\textbf{L’UTILITÉ DU WEB 2.0 DANS UNE ENTREPRISE
				}}
			
			Le Web 2.0 présente plusieurs avantages pour les entreprises :
			
			\begin{itemize}
				\item \textbf{Marketing et Communication}
	Les entreprises utilisent les réseaux sociaux comme Facebook, Instagram et Twitter pour interagir directement avec leur public. Par exemple, Nike a lancé des campagnes de marketing réussies en exploitant les médias sociaux pour promouvoir ses produits et engager sa communauté, démontrant ainsi l'efficacité du Web 2.0 dans le domaine du marketing (Smith, 2011).
		\item \textbf{Engagement Client}
	Les plateformes de service client en ligne comme Zendesk\footnote[7]{plateforme pour le service client.} ou les chats en direct offrent aux entreprises un moyen efficace d'interagir et de résoudre les problèmes de leurs clients en temps réel. Amazon, par exemple, utilise des outils de service client en ligne pour répondre aux demandes des utilisateurs, renforçant ainsi la fidélité de sa clientèle (Hunt, 2015).
	\item \textbf{Collaboration Interne}
	Des outils de collaboration en ligne comme Slack et Google Workspace sont utilisés par les entreprises pour faciliter le travail d'équipe à distance. Des entreprises comme Spotify tirent parti de ces plateformes pour permettre à leurs équipes réparties dans le monde entier de collaborer de manière efficace et en temps réel (Schoen, 2017).
	\item \textbf{Collecte de Données et Analytique}
	Les entreprises utilisent les données collectées à partir des interactions en ligne pour comprendre les comportements des consommateurs. Amazon et Netflix analysent les habitudes de leurs utilisateurs pour recommander des produits ou du contenu, améliorant ainsi l'expérience utilisateur et augmentant leurs ventes (Kumar et al., 2016).
	\item \textbf{Création de Contenu et Marketing de Contenu}
	Les blogs d'entreprise sont devenus des outils de marketing puissants. Des entreprises comme HubSpot utilisent des blogs pour fournir du contenu éducatif à leur public, attirant ainsi des prospects et renforçant leur autorité dans leur domaine (Halligan et Shah, 2010).
	\item\textbf{Crowdsourcing et Innovation Ouverte}
	Des entreprises comme LEGO utilisent le crowdsourcing pour impliquer les clients dans le processus d'innovation. Leur plateforme LEGO Ideas permet aux clients de proposer des idées de nouveaux produits, renforçant ainsi l'engagement des consommateurs et alimentant l'innovation (Afuh et al., 2018).
	\item \textbf{Commerce Électronique et Expérience Client}
	Les plateformes de commerce électronique comme Shopify et Magento permettent aux entreprises de créer des boutiques en ligne personnalisées. Des marques comme Zappos offrent des expériences client exceptionnelles via leur site web, facilitant ainsi les achats en ligne et fidélisant leur clientèle (Chaffey et al., 2019).
	\item \textbf{Gestion de la Marque et Feedback Client}
	Les entreprises surveillent les médias sociaux pour gérer leur réputation en ligne et recueillir les réactions des clients. Des entreprises comme Starbucks surveillent les commentaires des clients sur les réseaux sociaux pour répondre rapidement aux problèmes et maintenir une image positive de leur marque (Solomon et al., 2019).
       \end{itemize}
	Ces différentes utilisations du Web 2.0 démontrent la diversité des avantages qu'il offre aux entreprises, allant de la création de contenu et de l'innovation à l'amélioration de l'expérience client et de la gestion de la réputation en ligne. L'utilisation du Web 2.0 offre aux entreprises des opportunités significatives d'améliorer leurs stratégies de marketing, d'engagement client, de collaboration interne et d'analyse des données, démontrant ainsi son utilité et son impact sur la réussite commerciale dans le monde moderne.
	
	
	\section{\textbf{CONCLUSION}}
	En somme, le Web 2.0 représente une évolution majeure dans l'histoire d'Internet. Sa genèse, caractérisée par l'émergence de l'interaction bidirectionnelle, de la collaboration, et de la diversification des contenus en ligne, a marqué un tournant fondamental par rapport à l'ère précédente du Web 1.0. La comparaison entre les différentes phases, du Web 1.0 statique au Web 2.0 interactif, souligne l'importance de cette transition vers une ère où les individus sont devenus acteurs de la création et du partage de contenu. Les caractéristiques distinctives du Web 2.0, telles que l'interactivité, la collaboration et la personnalisation, ont profondément modifié la manière dont les entreprises interagissent avec leur public. L'utilité du Web 2.0 pour les entreprises s'est manifestée à travers divers domaines tels que le marketing, l'engagement client, la collaboration interne et l'analyse des données. Ces applications ont permis aux entreprises de s'adapter aux nouveaux paradigmes en ligne et d'améliorer leurs performances grâce à une interaction plus étroite avec leur clientèle. Alors que le Web 2.0 a joué un rôle crucial dans la transformation numérique, il s'inscrit également dans une continuité, préparant le terrain pour l'avènement du Web 3.0 avec ses avancées technologiques axées sur l'intelligence artificielle et l'Internet contextuel. Ainsi, le Web 2.0, par son impact sur la façon dont nous interagissons avec Internet et ses implications pour les entreprises, reste un pilier central de notre expérience numérique actuelle et un tremplin vers un avenir connecté, collaboratif et toujours plus innovant.
	\section{BIBLIOGRAPHIE }
	\begin{enumerate} 
		\item \cite{BernersLee1999}
		\item \cite{Anderson2007}
		\item \cite{McAfee2009}
		\item \cite{OReilly2005}
		\item \cite{Shirky2008}
		\item \cite{Tapscott2008}
	\end{enumerate}
	%% bibliographie
	\printbibliography
	
\end{document}